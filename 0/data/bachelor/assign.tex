% !Mode:: "TeX:UTF-8"
% 任务书中的信息
%% 原始资料及设计要求
\assignReq
{使用的原始文件为由tornado2.2编译的VxWorks5.5系统镜像,使用的}
{编译器为tornado2.2自带的gcc编译器。}
{使用的VxWorks5.5的BSP为使用tornado2.2编译的在Pentium下的BSP,}
{未经过修改。}
{对VxWorks5.5镜像的修改要求必须在编译完成之后进行。}
%% 工作内容
\assignWork
{修改VxWorks5.5操作系统的系统镜像文件,插入一段可执行的代码或数据,}
{从而在系统上电初始化之后,可以自动执行这一段代码。}
{代码要求能够劫持任何一个系统函数。即在该函数被其他函数调用时,}
{先执行被插入的代码。}
{}
{}
%% 参考文献
\assignRef
{[1] Cesare S. Unix ELF parasites and virus[M].[S.l.]: [s.n.] , 1998. }
{http://vxheavens.com/lib/vsc01.html.}
{[2] Ang Cui M. C., Stolfo S. J. When Firmware Modifications Attack: A Case}
{Study of Embedded Exploitation[R].[S.l.]: Columbia University, 2012.}
{[3] Committe T. Tool Interface Standard(TIS) Executable and Linking Formate}
{(ELF) Specification[M]. 1.2.[S.l.]: [s.n.] , 1995.}
{}{}
