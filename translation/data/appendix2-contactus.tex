% !Mode:: "TeX:UTF-8"
\chapter{实现文件监控功能的源代码}
\label{hook}

\begin{lstlisting}[
 language={C},
]
#include "stdio.h"
#include "hookFs.h"
#include "ioLib.h"
#include "private/dosFsLibP.h"
#include "vxWorks.h"
#include "Assert.h"
#include "hostLib.h"
#include "netDrv.h"

UINT32 read_ebp()
{
	UINT32 _ebp;
	__asm__ volatile("movl %%ebp, %0" : "=r"(_ebp));
	return _ebp;
}

time_t biostime()   
{
	struct tm ahora;
	unsigned char cHour, cMin, cSec; 
	unsigned char cDay, cMonth, cYear; 
  
	sysOutByte(0x70,0x00/*second*/); 
	cSec = sysInByte(0x71); 
	ahora.tm_sec = (cSec&0x0F) + 10*((cSec&0xF0)>>4); 
  
	sysOutByte(0x70,0x02/*minut*/); 
	cMin = sysInByte(0x71); 
	ahora.tm_min = (cMin&0x0F) + 10*((cMin&0xF0)>>4); 
  
	sysOutByte(0x70,0x04/*hour*/); 
	cHour = sysInByte(0x71); 
	ahora.tm_hour = (cHour&0x0F) + 10*((cHour&0xF0)>>4); 
  
	sysOutByte(0x70,0x07/*day*/); 
	cDay = sysInByte(0x71); 
	ahora.tm_mday = (cDay&0x0F) + 10*((cDay&0xF0)>>4); 
  
	sysOutByte(0x70,0x08/*month*/); 
	cMonth = sysInByte(0x71); 
	ahora.tm_mon = (cMonth&0x0F) + 10*((cMonth&0xF0)>>4) - 1; 
  
	sysOutByte(0x70,0x09/*year*/); 
	cYear = sysInByte(0x71); 
	ahora.tm_year = 100 + (cYear&0x0F) + 10*((cYear&0xF0)>>4); 
  
	return mktime(&ahora); 
}

void get_time(char datetime[])
{
	int res; 
	struct timespec ts; 
	struct tm daytime; 
	time_t stime; 
  
	ts.tv_sec = biostime(); 
	ts.tv_nsec = 0; 
	res = clock_settime(CLOCK_REALTIME, &ts); 
	
	stime = time(NULL); 

	daytime = *localtime(&stime);
	strcpy(datetime, asctime(&daytime));
}

#define FUNC_HOOK_FS(name) \
  void hookFs##name() \
  {                                 \
	UINT32 *_ebp; \
	UINT32 arg; \
	/*UINT32 arg0, arg1, arg2, arg3, arg4; */\
	char nameBuf[MAX_PNAME_LEN]; \
	PathNode *pn; \
	DOS_FILE_DESC_ID pfd; \
	if(!pHead) \
	{ \
		printf("NOTIFY NOT INIT.\n"); \
		return; \
	} \
	pn = pHead->next; \
	if(!pn) \
		return; \
	memset(nameBuf, 0, sizeof(nameBuf)); \
	_ebp = (UINT32 *) read_ebp(); \
	_ebp = _ebp[0]; \
	/* arg0 = _ebp[2];\
	arg1 = _ebp[3];\
	arg2 = _ebp[4];\
	arg3 = _ebp[5];\
	arg4 = _ebp[6];*/\
	if( !strcmp(#name, "Open") || !strcmp(#name, "Create")  ) { \
		arg = _ebp[4]; \
		strcpy(nameBuf, arg); \
	} \
	else if (!strcmp(#name, "Delete")) \
	{ \
		_ebp = (UINT32 *) read_ebp(); \
		arg = _ebp[2];\
		strcpy(nameBuf, arg); \
	} \
	else if( !strcmp(#name, "Close") ) \
	{ \
		arg = _ebp[2]; \
		pfd = (DOS_FILE_DESC_ID)arg; \
		dosChkBuildPath(pfd); \
		strcpy(nameBuf, pfd->pVolDesc->pChkDesc->chkPath); \
	} \
	else { \
		/* Get the pFd */ \
		_ebp = (UINT32 *) read_ebp(); \
		/*arg0 = _ebp[2];\
		arg1 = _ebp[3];\
		arg2 = _ebp[4];\
		arg3 = _ebp[5];\
		arg4 = _ebp[6];*/\
		arg = _ebp[2]; \
		pfd = (DOS_FILE_DESC_ID)arg; \
		dosChkBuildPath(pfd); \
		strcpy(nameBuf, pfd->pVolDesc->pChkDesc->chkPath); \
	} \
	while(pn) \
	{ \
		if(!strcmp(nameBuf, pn->pName) || !pathcmp(pn->pName, nameBuf)) \
		{ \
			char datetime[64]; \
			char result[MAX_PNAME_LEN + 80]; \
			memset(datetime, 0, sizeof(datetime)); \
			get_time(datetime); \
			memset(result, 0, sizeof(result)); \
			strcpy(result,#name); \
			strcat(result,"\t"); \
			strcat(result,nameBuf); \
			strcat(result,"\t"); \
			strcat(result,datetime); \
			strcat(result,"\r\n\0"); \
			printf("%s\n",result); \
			break; \
		} \
		pn = pn->next; \
	} \
}

FUNC_HOOK_FS(Open);
FUNC_HOOK_FS(Create);
FUNC_HOOK_FS(Read);
FUNC_HOOK_FS(Write);
FUNC_HOOK_FS(Close);
FUNC_HOOK_FS(Delete);
\end{lstlisting}
