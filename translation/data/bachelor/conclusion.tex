% !Mode:: "TeX:UTF-8"
\chapter*{结论}
\addcontentsline{toc}{chapter}{结论}

\begin{itemize}
  \item 取得的成果
\end{itemize}

经过分析和实验证明,
通过在nop指令串中写入代码,
可以将代码插入到VxWorks二进制镜像文件中。
劫持方法上,使用间接跳转法进行劫持,
即利用一段插入代码proxy作为中间代码。
可以简化劫持流程;
另一方面,由于无需修改插入函数,
也很大程度上方便了插入函数的编写。


利用选定的插入和劫持方案,
在VxWorks中实现了一个类似Linux下inotify的文件监控机制——mini-notify。
该机制能够捕获系统中所有的文件操作,
并输出操作的文件路径等信息。
从而证明了插入与劫持技术在VxWorks下的可行性。

\begin{itemize}
  \item 尚存的不足
\end{itemize}

相对于在Unix/Linux下丰富的ELF插入技术,
本文经分析认为只有利用nop串插入在VxWorks中
是可行的。没有找到其他合适的代码插入方式。

VxWorks系统已经实现了动态插入一个模块的功能,
类似于Linux下的动态链接程序,
本文缺少对这一方面的分析和研究。


\cleardoublepage
