% !Mode:: "TeX:UTF-8"
\chapter{介绍}

这篇文章解释了一个主要的计算机安全性问题,
即病毒。病毒的有趣之处在于它能依附于其
他的程序并使它们也成为病毒。考虑到现代
计算机系统的广泛共享,携带着木马程序
\upcite{1,20}
的病毒的威胁越来越值得注意。尽管在阻止信
息非法传播
\upcite{4,7}
的执行政策上有相当多的工
作已经完成,并且许多系统已经采用它们
来避免这种类型的攻击
\upcite{12,19,21,22}
,但
是在如何保持信息进入一个领域而不导致
破坏方面的工作却很少
\upcite{5,18}。
在计算机系统中有很多不同类型的信息路径,其中一些
是合法授权的,另一些可能是隐秘的
\upcite{18},
这常常被用户所忽略。在这篇文章中我们将忽略隐秘性的信息路径。

一般设施提供正确的保护方案
\upcite{9},
但是
它们依赖于只对正在进行的攻击能有效防
御的安全性政策。就算是一些相当简单的防
护系统也不能被证明是“安全的”
\upcite{14}。
防止拒绝服务需要停止程序的检测
\upcite{11}。给系统中
的信息流进行精确标记的问题已被证明
是一个NP-完全问题。对于在用户之间的不可信信息
的传播,进行某种方式的防卫,这已经被
测试过了
\upcite{24},但是一般
依赖能力来提高程序正确性,这也是一个
众所周知的NP完全问题。

施乐公司的蠕虫程序已经说明,该程序可
以通过网络繁殖,并且可以偶然地造成
设备无法服务。在后来的变种中,一个“核心
战争”的游戏中,能够让两个程序互相争斗。一
些针对这一主题的,被匿名作者发布的其
他的变种,是基于在程序之间进行的
夜间游戏的上下文中的。术语病毒也被
用于结合丰APL的作者的地方一般开始时调用每个函数反
过来调用一个预处理器,以增加默认APL翻译。

一个广泛传播的安全问题的潜在威胁被分析了\upcite{15},对于政府、金融、商业和学术设施的危害是严重的。另外,这些机构倾向于使用特别的保护机制来应对这些特有的威胁,而不是研究全面而彻底的理论上的保护机制\upcite{16}。当前的军队保护系统,很大程度上依赖隔离机制\upcite{3};然而,新的系统开始使用多层的用法\upcite{17}。没有一个发布出来的被建议使用的系统可以定义或者部署一种方法,用来阻止一种病毒。


这篇文章中我们提出了一个防止计算机病毒的新问题。首先我们检测了病毒的感染特性,并且展示了共享信息的传递闭包可以被感染。当和木马一起使用的时候,可以导致设备广泛停止服务以及数据的非授权的操作。一些对计算机病毒的实验结果表明,病毒对于普通的和高安全性的操作系统都是一个强大的威胁。传播的途径、信息流的传递以及信息解释的一般性是防止计算机病毒的关键特性,这些特性将在下面一一解释。分析表明,有潜力阻止病毒攻击的系统具有有限的传递性和有限的共享性,或者完全不能共享也不具有一般的信息解释(图灵能力)。只有前一种情形对于现代社会具有研究的意义。一般来说,使用先验和运行时分析对病毒的防护的研究是不可判定的,并且没有防护,治愈是很难或者无法实现的。


我们检查了一些被提出来的对策,这些对策都针对特定的情景,来对病毒的特质进行就事论事的分析。限制传播系统被认为是值得期待的,但是精确的部署是很棘手的,并且模糊的政策一般来说会导致有用的系统随着时间越来越少。系统范围内的病毒抗体的使用也被测试了,但普遍来说,还是要依赖针对特定棘手问题的解决方案。
	


综上可知对计算机病毒是一个非常重要的研究领域,因为它和其他领域的潜在性应用。现代系统对于病毒攻击提供很少或者没有提供防护,并且当代唯一有效的“安全”措施就是隔离了。












