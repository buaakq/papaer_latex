% !Mode:: "TeX:UTF-8"
\chapter{绪论}

\section{课题背景与意义}

代码插入意味着通过将二进制机器码注入到可执行文件中,
使这段代码随着程序执行被映射到内存空间。
劫持则是指在程序合适的位置插入跳转指令,
从而在特定的时机获取控制权,使插入代码被执行。
二者可以合称为二进制代码修改技术。
%%
二进制代码插入与劫持技术是编写病毒、调试和破解闭源系统及软件、
入侵检测和预防的基础技术之一\upcite{cerberus}。
%%
利用二进制代码插入编写的UNIX平台上的病毒已经不是新鲜事,
早在1998年,文献\cite{silvio}就给出了一个原始的UNIX病毒的原型。

由Wind River公司开发的VxWorks操作系统是
国内外最广泛使用的具有极高实时性和可靠性的嵌入式操作系统之一。
很多嵌入式设备都使用它,例如打印机。
%%
文件格式上,VxWorks系统使用了与UNIX系统相同的ELF
(Executable and Linkable Format,可链接执行格式)文件,
这一方面为UNIX平台的病毒在VxWorks上的移植和蔓延提供了方便,
另一方面,我们也可以使用类似UNIX上的二进制代码技术
对其进行二进制层面的破解、调试和攻击。
%%
最近,一项针对惠普打印机的攻击实验\upcite{printme}已经证实,
当有可利用的漏洞用于获取嵌入式设备固件并修改之的时候,
二进制层面的代码修改技术将大有用武之地。


鉴于此,
本课题在整理和分析已有的UNIX下ELF文件插入与劫持技术的基础上,
寻找合适的技术并设计方案,
实现对于VxWorks 5.5操作系统映像文件的二进制代码插入操作,
并修改指令执行流,在特定的位置实现劫持。
目的是在不影响系统正常运行的前提下,
系统启动完成后,在合适的时机获取控制权,
执行我们事先插入的代码。

\section{国内外研究现状}

\subsection{ELF文件代码插入技术}

一个最简单的代码插入技术是利用ELF可执行文件
中存在的nop指令片段\upcite{heike,infelf}。
这些nop指令通常是编译器为了对齐函数起始地址而留下的。
在这些零碎的片段中写入二进制代码,可以轻松地将
代码植入目标程序中。

二进制代码插入技术也用于制作病毒,Silvio Cesare在\cite{silvio}
首次提出在UNIX中通过静态修改ELF可执行文件来注入代码。
插入代码的映射到内存的位置,
是ELF文件中的可加载段之间存在的
大量的空闲地址空间。
在插入代码的同时,需要小心地修改一些ELF
的数据结构,从而让文件“接纳”这一插入。
这一方法后来被认为是比较合适的插入方式,
并被许多针对ELF乃至Linux内核的工作所采用
和拓展,
例如\cite{simple,prototype,subversive,cerberus,sharelib}。


在Linux对ELF文件进行插入的
另一种思路是利用进程地址
空间中未被使用的一部分虚拟地址,
例如8048000h之前的地址\upcite{cerberus}。
这一方法相对于在段间插入来说,
更加直观明了,且插入的代码量也更大。
更有一个自动化的插入工具ELFsh\upcite{elfsh}
也采用了这一方法。
然而,这一办法较为复杂的一点是需要建立新的
节和段,并建立它们的映射关系。
这对于病毒制作者来说当然是不能接受的。

\subsection{ELF可执行文件劫持技术}

文献\cite{silvio}中提到的病毒使用的劫持方式是
修改函数的入口点,即修改ELF文件头entrypoint字段中的地址。
后来的一些病毒模型中大都也采用相同的办法,如
例如\cite{simple,prototype}。

与修改程序的入口点类似,
在程序体中插入跳转指令同样可以改变
指令执行流。然而出于不影响程序正常执行的目的,
往往需要在插入代码中进行一些工作,
例如保存和恢复寄存器等。

随着更多的程序开始使用动态链接机制,
人们开始从动态链接的原理和数据结构中寻找突破口。
利用动态链接机制最大的好处是,
不一定非要在程序的开始处执行被插入的代码,
而是可以劫持任何一个特定的函数。
文献\cite{sharelib}较早提出了一种在插入的代码中
修改plt(Procedure Linkage Table,过程链接表)
的机制,一旦成功,
程序每次调用某个动态库函数(例如printf),
在进入函数体之前,会先执行一段事先插入的代码。
采用类似的办法,文献\cite{modern}进行了大幅改进,
不需要在程序执行之前修改磁盘的文件,
即程序已经开始运行,通过修改内存也能实现劫持。


\subsection{其他ELF文件修改技术}
在ELF文件中插入代码无论是否利用动态链接机制,
最终都可以归结为绕过链接的一种文件合并。
无论插入代码的技术多么先进和隐蔽,
能否在被插入代码体中完成一定任务,
而不是打印一行“hello world!”就归还控制权,
显得更为重要,
特别是在插入代码量有一定限制的情况下。
%%
然而我们的插入工具始终不能像编译器和链接器那样,
在ELF的相应数据结构中植入对动态库函数的引用,
因此插入的代码往往只能使用系统调用。
针对这一问题,文献\cite{subversive}提出了在插入代码中,
利用显式运行时链接机制,
显式地要求动态链接器为我们打开所需要的共享库。
从而帮助我们完成任务。
这也说明,动态链接机制不但帮助程序节省了空间,
也让在夹缝中生存的病毒有了更多可利用的资源。

为了把ELF文件修改操作接口化,
一个专门针对ELF文件修改的工具ELFsh\upcite{elfsh}被开发了出来。
后续的一些工作如\cite{cerberus}开始充分利用它的强大功能。
本课题的研究不但使用了它的接口,
也借鉴了其源代码中包含的一些技术。


\subsection{嵌入式系统中的二进制代码技术}

逆向工程师一个乐此不疲的目标就是破解各种嵌入式设备固件,
例如打印机\upcite{printme},Apple公司的产品\upcite{apple},
甚至对硬盘也不放过\upcite{disk}。


在ELF代码插入技术逐渐被完善甚至工具化的同时,
所有使用了ELF文件格式的平台都应当保持警惕。
%%
VxWorks在工业和国防等领域有着广泛的应用\upcite{vxworks1},
因此很可能成为潜在的攻击对象。
%%
一个典型的例子是\cite{printme},在利用惠普打印机固件
更新漏洞的基础上,
通过网络对打印机的VxWorks的固件进行了修改。
当打印机再次工作时,
攻击者们就可以通过网络得到打印机的所有工作信息,
比如打印文档的内容。
%%
当然,与插入和修改系统镜像的技术相比,找到能利用的漏洞
同样重要,甚至是先决条件,因为毕竟不是每个人都能直接读写设备固件。
%%
但我们在此更关注注入技术:即攻击者如何修改了VxWorks镜像,
让打印机能够为它工作,并且不出现异常。
一旦这项技术在特定场景下被实例化,
在物理隔离被破坏的基础上,哪怕是一瞬间,
也可能意味着你的嵌入式设备将会成为你的间谍。

\section{课题研究目标与内容}

本课题的目标归纳为以下两点:

\begin{itemize}
  \item 寻找合适的办法并设计方案,在VxWorks系统镜像中插入代码。
  \item 修改VxWorks系统镜像中的指令,使插入的代码在合适的时机得以被执行。
\end{itemize}

为了达成以上的两点目标,
本课题主要进行了以下几个方面的工作:

\begin{itemize}
  \item 分析现有的Linux下针对ELF的插入技术和劫持技术,分析总结其特点。
  \item 分析VxWorks可能影响代码插入和劫持的相关特性,
并选择合适的插入和劫持方案,理论上分析其可行性。
  \item 进行一个实例(mini\_notify)的研究,验证代码插入和劫持方案
在VxWorks系统中的可行性。
\end{itemize}

\section{论文组织结构}

第二章全面论述针对Linux下ELF的文件插入和代码劫持技术。
并通过实例分析其原理和缺陷。

第三章考察VxWorks中与代码插入有关的特性,
进而确定合适的ELF文件的插入方式。

第四章考察VxWorks中与程序劫持有关的特性,
确定合适的劫持方案并证明其可行性。

第五章借助前两章确定的插入和劫持方案,
实现一个小型的VxWorks文件监控系统——mini\_notify,
证明相应方案的可行性。

最后进行简要总结。

%%%\section{相关信息}
%%%
%%%\subsection{模板维护及联系方式}
%%%\begin{tabular}{ll}
%%%    \multicolumn{2}{l}{北航开源俱乐部 BeiHang OpenSource Club (BHOSC)} \\
%%%    GoogleGroup & \url{https://groups.google.com/d/forum/BHOSC/} \\
%%%    Github      & \url{https://github.com/BHOSC/} \\
%%%    IRC         & \#beihang-osc @ FreeNode
%%%\end{tabular}
%%%
%%%\subsection{代码托管及相关页面}
%%%\begin{itemize}
%%%    \item 毕业设计论文模板代码
%%%    \item[] \url{https://github.com/BHOSC/BUAAthesis/}
%%%    % TODO(huxuan): Others pages related to BUAAthesis
%%%    % \item 软件学院本科毕设答辩演示模板
%%%    % \item[] \url{https://github.com/huxuan/latex\_buaasoft\_bachelor\_slide}
%%%    % \item 研究生毕设综述报告和开题报告模板
%%%    % \item[] \url{https://github.com/JosephPeng/ZongshuKaiti\}
%%%\end{itemize}
%%%
%%%\subsection{贡献者}
%%%\begin{tabularx}{\textwidth}{@{\hspace{2em}}ll}
%%%    \href{https://github.com/JosephPeng/}{Joseph \footnote{目前的维护者}} &
%%%    \href{mailto:mrpeng000@gmail.com}{mrpeng000@gmail.com} \\
%%%    \href{http://huxuan.org/}{huxuan \textsuperscript{1}} &
%%%    \href{mailto:i@huxuan.org}{i@huxuan.org} \\
%%%\end{tabularx}
%%%
%%%\subsection{项目协议}
%%%本项目主要遵从以下两套协议
%%%\begin{itemize}
%%%    \item \href{http://www.gnu.org/licenses/gpl.txt}
%%%        {GNU General Public License (GPLv3)}
%%%    \item \href{http://www.latex-project.org/lppl.txt}
%%%        {\LaTeX{} Project Public License (LPPL)}
%%%\end{itemize}
%%%使用前请认真阅读相关协议,详情请见项目代码根目录下的 LICENSE 文件
%%%
%%%\section{免责声明}
%%%本模板为编写者依据北京航空航天大学研究生院及教务处出台的
%%%《北京航空航天大学研究生撰写学位论文规定(2009年7月修订)》和
%%%《本科生毕业设计(论文)撰写规范及要求》编写而成,
%%%旨在方便北京航空航天大学毕业生撰写学位论文使用。
%%%
%%%如前所述,本模板为北航开源俱乐部\LaTeX{}爱好者依据学校的要求规范编写,
%%%研究生院及教务处只提供毕业论文的写作规范,目前并未提供官方\LaTeX{}模板,
%%%也未授权第三方模板为官方模板,故此模板仅为论文规范的参考实现,
%%%不保证格式能完全满足审查老师要求。任何由于使用本模板而引起的论文格式等问题,
%%%以及造成的可能后果,均与本模板编写者无关。
%%%
%%%任何组织或个人以本模板为基础进行修改、扩展而生成新模板,请严格遵守相关协议。
%%%由于违反协议而引起的任何纠纷争端均与本模板编写者无关。
%%%
%%%\section{版本历史}
%%%\begin{itemize}
%%%    \item[1.0] 2012/07/24 已完成大体功能,说明文档和细节方面还有待完善。
%%%    % “a.b”为版本号,b为奇数时为内测版本,为偶数时为发行版。
%%%\end{itemize}
