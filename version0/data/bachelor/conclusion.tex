% !Mode:: "TeX:UTF-8"
\chapter*{结论}
\addcontentsline{toc}{chapter}{结论}

\begin{itemize}
  \item 取得的成果
\end{itemize}

本文通过实例证明,
现有的针对Unix/Linux下的ELF文件代码插入技术,
可以在VxWorks5.5系统中得以应用。

本文经过分析,
认为有两种方法可以利用,
分别是使用在nop串中或者数据段写入代码。
利用这两种方法插入代码,
并修改相应的跳转,
可以劫持VxWorks中的任何一个函数。

实际的劫持跳转流程中,
本文提出了一个“中间层”的思想,
即利用一段插入代码proxy作为跳转,
利用proxy实现堆栈的保护,
并在proxy中再调用相应的插入函数,
从而把插入函数隔离了出来,
使得插入工作相对于插入函数的编写是透明的。

本文还实现了一个插入的例子,
可以劫持dosFs文件系统的函数,
从而达到监控文件系统操作的目的。

\begin{itemize}
  \item 尚存的不足
\end{itemize}

相对于在Unix/Linux下丰富的ELF插入技术,
本文在VxWorks下并没有找到较多合适的插入技术,
最终的插入都仅使用在nop串中插入代码和直接覆盖数据节来完成。
而由于覆盖数据节并不具有通用性,
实际上只有在nop串插入代码一种方法真实可行。

VxWorks系统已经实现了动态插入一个模块的功能,
类似与Linux下的动态链接程序,
本文缺少对这一方面的分析和研究。


\cleardoublepage
